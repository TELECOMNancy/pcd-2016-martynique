\chapter{Mercredi}


Les objectifs de cette journée sont les suivants:

\begin{itemize}
	\item Pouvoir regarder un vidéo offline (en retard)
	\item Télécharger une vidéo (en retard)
	\item Sauvegarder des vidéos favoris.
	\item Pouvoir créer une playlist
	\item Améliorer l'éxperience utilisateur avec des thumbnails avancées.
\end{itemize}

La sauvegarde des favoris nous mènent naturellement vers un SGBD. Ce SGBD pourra également nous servir pour le module de suggestion de vidéos.

Nous rencontrons nos premières difficultés pour la conception. Nous commencons à duppliquer du code et des design pattern peuvent être appliqués dans certaines parties du projet. Retravailler la conception du projet nous ferait perdre beaucoup de temps. C'est pourquoi, nous décidons de continuer sur cette voie mais en migrant progressivement vers la nouvelle conception.

Point positif, une avancé importante a été apportée au niveau de l'interface homme-machine. Nous nous inspirons directement de l'application android \textit{Youtube}. De plus, les vidéos favorites de l'utilisateur sont persistés en base de données.


Très grande avancée dans le projet (player design, suggestion, player, base de données, refonte la conception). Le projet est sur une bonne voie